\documentclass{article}
\usepackage{amsmath}
\usepackage{amsthm}
\usepackage{amssymb}
\usepackage{comment}
\usepackage{mathtools,xparse}
\usepackage[mathscr]{euscript}
\let\euscr\mathscr

\DeclarePairedDelimiter\abs{\lvert}{\rvert}%
\DeclarePairedDelimiter\norm{\lVert}{\rVert}%

\begin{document}
\title{MAT300: Homework 4}
\author{
	Joseph Bryan IV\\
	Benjamin Friedman\\
	Koranis (Sandy) Tanwisuth
}

\maketitle
	\begin{flushleft}
		\textbf{pp. 62-64: 3.2.6, 3.3.2, Q2P 3}
	\begin{enumerate}
		\setcounter{enumi}{2}
		\item Induction \\
		
		\textbf{\textit{3.2.6}}
		\begin{flushleft}
			Let $x \neq 1$ be a real number. For all $n \in \mathbb{N}$. 
			\begin{equation}
				\frac{x^n-1}{x-1} = x^{n-1}+x^{n-2}+...+x^2+x+1
			\end{equation}
			\\
			\begin{proof}
				By induction, let \\
				\begin{align*}
					P(n): \frac{x^n-1}{x-1} & = x^{n-1}+x^{n-2}+...+x^2+x+1 \\
					& = \sum\limits_{i=0}^{n-1} x^i
				\end{align*} \\
				\textbf{Base Case ($n = 1$):}
				\\ When $P(n=1)$, the left hand side of the equation equals to \[\frac{x^1-1}{x-1} = \frac{(x-1)}{(x-1)}  = 1 \]
				while the right hand side equals to $x^0 = 1$. Thus, both sides are equal and $P(n)$ is true for $n=1$.
				\\
				\pagebreak
				\textbf{Induction Step:}
				\\
				Let $k$ be an arbitrary but fixed element in $N$ and suppose that $P(k)$ is true, i.e. $\frac{x^k-1}{x-1} = \sum\limits_{i=0}^{k-1} x^i$. Then
				\begin{align*}
					\frac{x^{k+1}-1}{x-1}
					& = \sum\limits_{i=0}^{k-1} x^i + x^{(k+1)-1} \\
					& = (x^{k-1}+x^{k-2}+...+x^2+x+1) + x^k, \thickspace by \thickspace induction \thickspace hypothesis \\
					& = x^k+x^{k-1}+x^{k-2}+...+x^2+x+1 \\
					& = \sum\limits_{i=0}^{k} x^i \\
					& = \sum\limits_{i=0}^{(k+1)-1} x^i 
				\end{align*}
				Thus, $P(n)$ holds for $n=k+1$. 
				By the principle of mathematical induction, $P(n)$ is true for all $n \in \mathbb{N}$.
			\end{proof}
			
		\end{flushleft}
		
		\textbf{\textit{3.3.2}}
		\begin{flushleft}
		Use complete induction to prove that every natural number can be written as the sum of distinct powers of two.
			\\
			\begin{proof}
				By complete induction, \\
				\[P(n): n \geq 1\thickspace is\thickspace the\thickspace sum\thickspace of\thickspace distinct\thickspace power\thickspace of\thickspace two\]
				\textbf{Base Case ($n = 1$):}
				\\ \vspace{2mm}
				When $n = 1$, $P(1)$ equals to $2^0$. Thus, $P(n)$ holds for $n=1$.
				\\ \vspace{2mm}
				\textbf{Induction Step:}
				\\ \vspace{2mm}
				Suppose $n \geq 1$ be an arbitrary but fixed element in $N$ and suppose that $P(k)$ is true for all $k \leq n$ by induction hypothesis. Then, \textbf{consider $n+1$},\\
				\vspace{2mm}
				If $n+1$ is even, then there exists an integer $k \leq n$ such that $(n+1) = 2k$. By induction hypothesis, $k$ is the sum of distinct power of $2$ and hence so is $(n+1)$.
				\\ \vspace{2mm}
				If $n+1$ is odd, then $n$ is even, and by induction hypothesis is a sum of distinct powers of $2$, which does not contain a summand of $2^n=1$. Therefore, $(n+1) = n + 2^0$ is a sum of distincts powers of 2. 
				Thus, $a_n$ holds for $n=k+1$.
				\\ \vspace{2mm}
				By the principle of mathematical induction, $P(n)$ is true for all $n \in \mathbb{N}$.
			\end{proof}
		\end{flushleft}
		
		\textbf{\textit{Q2P: 2}}
		\begin{flushleft}
			Formulate and prove a generalization of the Principle of Mathematical Induction in which the base case is an arbitary $m \in \mathbb{Z}$.
			\\
			\begin{proof}
				By complete induction, \\
				\[P(n): n \geq 1\thickspace is\thickspace the\thickspace sum\thickspace of\thickspace distinct\thickspace power\thickspace of\thickspace two\]
				\textbf{Base Case ($n = 1$):}
				\\ \vspace{2mm}
				When $n = 1$, $P(1)$ equals to $2^0$. Thus, $P(n)$ holds for $n=1$.
				\\ \vspace{2mm}
				\textbf{Induction Step:}
				\\ \vspace{2mm}
				Suppose $n \geq 1$ be an arbitrary but fixed element in $N$ and suppose that $P(k)$ is true for all $k \leq n$ by induction hypothesis. Then, \textbf{consider $n+1$},\\
				\vspace{2mm}
				If $n+1$ is even, then there exists an integer $k \leq n$ such that $(n+1) = 2k$. By induction hypothesis, $k$ is the sum of distinct power of $2$ and hence so is $(n+1)$.
				\\ \vspace{2mm}
				If $n+1$ is odd, then $n$ is even, and by induction hypothesis is a sum of distinct powers of $2$, which does not contain a summand of $2^n=1$. Therefore, $(n+1) = n + 2^0$ is a sum of distincts powers of 2. 
				Thus, $a_n$ holds for $n=k+1$.
				\\ \vspace{2mm}
				By the principle of mathematical induction, $P(n)$ is true for all $n \in \mathbb{N}$.
			\end{proof}
		\end{flushleft}
		
		\end{enumerate}
	\end{flushleft}
\end{document}