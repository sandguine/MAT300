\documentclass{article}
\usepackage{amsmath}
\usepackage{amsthm}
\usepackage{amssymb}
\usepackage{comment}
\usepackage{mathtools,xparse}
\usepackage[mathscr]{euscript}
\let\euscr\mathscr

\DeclarePairedDelimiter\abs{\lvert}{\rvert}%
\DeclarePairedDelimiter\norm{\lVert}{\rVert}%


\begin{document}
\title{MAT300: Homework 6}
\author{
	Joseph (Shep) Bryan IV \\
	Benjamin (Ben) Friedman\\
	Koranis (Sandy) Tanwisuth
}

\maketitle
	\begin{flushleft}
		\textbf{p.68: 4.1.10, pp.75-76: 4.2.15, pp.96-97: 2, 3c, 5, 10, 11}

		\textbf{\textit{4.1.10}}
		\begin{flushleft}
			Indicate whether the following relations are reflexive, symmetric, antisymmetric, or transitive.
			\begin{enumerate}
				\item $A = \{p: p$ is a person in Alaska\}. $x$~$y$ if $x$ is at least as tall as $y$.
				\\
				Let A be ab arbitary but fixed set as given. Let R be a relation where By definition of reflexive, a relation A is reflexive if every element of A is related to itself (p.68: 4.1.8).  Since 
				
			\end{enumerate}
		\end{flushleft}
		
		\textbf{\textit{4.2.15}}
		\begin{flushleft}
			test
		\end{flushleft}
		\begin{enumerate}
			\setcounter{enumi}{1}
			\item Q2
		
			\item Q3
		
			\setcounter{enumi}{4}
			\item Q5
		
			\setcounter{enumi}{9}
			\item Q10
		
			\item Q11
	
		\end{enumerate}
	\end{flushleft}
\end{document}